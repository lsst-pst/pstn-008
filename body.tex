\section{Overview}

The purpose of the AOS is to optimize the image quality across the field by controlling the surface figures
of the mirrors (M1M3 and M2) and to maintain the relative position of the three optical systems
(M1M3 mirror, M2 mirror and the camera).
The mirror surfaces are adjusted by means of figure control actuators that support the mirrors.
Although they are commonly called ``figure control actuators'' the majority of their load is utilized to
support the mirrors against the forces of gravity. The relative rigid body positions of M1M3, M2 and
the camera are controlled through the M2 and Camera hexapods.
The M2 and the Camera are positioned relative to the M1M3.


The AOS is principally operated off of a Look-Up Table (LUT).
The LUT provides open loop, near optimum values for all actuator forces and hexapod positions.
The LUT values vary principally with elevation angle and secondarily with temperature.

Although the LUT values are near optimum, as a result of non-repeatable effects,
they are inadequate to reach the image quality requirements. These effects include temperature
distributions on the telescope structure and mirrors along with wind loading and hysteresis.
Corrections are added to the LUT values based on wavefront measurements from the wavefront sensors
in the camera's focal plane.

The position of the mirrors relative to their mirror cells is controlled with hard points.
The proper load is maintained in the hard points by applying distributed loads through the figure control
actuators in the form of correction added to the LUT values. Since the wavefront correction
is intended to bend the mirror and apply no net forces and the force balance correction is intended to
produce specific sets of net forces without bending the mirror, the two systems are compatible and
can operate simultaneously.

The force balance offset is added along with the wavefront correction to the LUT values.
To allow more rapid responses, the force balance is accomplished directly by the mirror support control systems.
This allows the force balance to accommodate dynamic loading and the quasi-static component of wind loading.

Control strategy and image quality error budget

Operational considerations

\section{Active Optics Hardware Performance}

Short summaries only. More details presented in the another construction paper - ``performance of the Delivered LSST System''.

M1M3 mirror and cell assembly: Mirror Lab testing, M3 in-situ

M2 mirror and cell assembly: testing at Harris and summit

Hexapods and camera rotator: Moog testing and summit re-verification

mention IOTA?

Alignment System performance

\section{Curvature Wavefront Sensing}

Short summary only, refer to Applied Optics paper.

Show real example, including image processing, donuts, cwfs output.

\section{State Estimation and Control}

bending modes, optical sensitivity matrix, and their evolutions

State estimator, optimal controller, matrix truncation etc.

discuss field variations of the Zernikes, and compare to aberration theory predictions (Schechter)

\section{Look-Up Table Construction and Operation}

Started from FEA, refined by optical testing, IOTA, ComCam, eventually LSSTCam

\section{Image Quality and Ellipticity Performance}

Give quick examples from ComCam and LSSTCam.

Refer to ``performance of the Delivered LSST System''.

\section{Simulations and Early Testing with other Telescopes}

Role of simulations in system development.

Reconciliation of simulated performance vs measured performance.

\section{Conclusions}

The AOS has been working reliably since xxx 2021.
